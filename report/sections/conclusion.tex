\section{Conclusions and Discussion}
To summarize, we have found that the mental distresses of the participants in both the intervention group and the control group decrease significantly over time, although the magnitude of decrease may be small on a monthly level. We have also found that mental health intervention does not seem to have a significant effect in reducing the level of mental distress. By comparing the results from the linear mixed-effects models against those from the GEE models, we have confirmed that the conclusions are reliable despite minor violations of the linear mixed-effects model's assumptions. Through addressing the missing data using multiple imputation, we have verified that the conclusions we have drawn are not overly sensitive to the missing data.\\\\
In terms of limitations of the study, it is firstly important to note that there is no information on how the participants were selected. While the treatment were assigned randomly to each participant, it is not clear whether the participants were also randomly selected. If the group of the participants were indeed not representative of the overall population, the presented analysis results may have been biased. Secondly, we have no knowledge of why some of the data are missing. Although we have used the multiple imputation method to address this issue, it is important to know that the missing value predictions were based on the other observed data. There is then an implicit assumption that the missing data depend on the other variables recorded in the study. If this assumption does not hold, the analysis results may still be biased.\\\\
For future studies, if the selection of participants and why some of the data are missing can be better documented, we will be more confident that the conclusions drawn can be more generally applicable to the overall population of interest.