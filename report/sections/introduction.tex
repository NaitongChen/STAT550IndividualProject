% !TEX root = ../main.tex

% Background section

\section{Introduction}\label{sec:introduction}
Nowadays, mental distress has become a common social issue and is affecting many people's quality of life. According to the American Psychiatric Association, up to nineteen (19) percent of adults in the United States experience some degree of mental illness \cite{apastat}. Amid the COVID-19 pandemic, the mental well-being of the general population has received an unprecedented amount of attention \cite{twenge2020mental}. It is therefore important to study the effectiveness of mental health interventions on reducing the level of mental distress.\\\\
This study investigates changes in participants' mental distress over time in an intervention group and a control group. Specifically, we try to answer whether the participants' mental distresses decrease over time and whether the mental health intervention is effective in reducing the level of mental distress.\\\\
This report begins by introducing the data recorded for this study and drawing preliminary conclusions through some explorative data analysis. The reliability of these preliminary conclusions are then checked through some model-based statistical analysis. Finally, we provide a discussion on the implication of the findings in this study. The R code used for this report can be found at \url{https://github.com/NaitongChen/STAT550IndividualProject}.
% ...